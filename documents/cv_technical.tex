\documentclass[12pt]{article}

\usepackage{graphicx}
\graphicspath{{figs/}}

\usepackage{color}
\newcommand{\comment}[1]{{\color{red} [#1]}}
\newcommand{\highlight}[1]{{\color{green} #1}}

\newcommand{\given}{\,|\,}
\newcommand{\setof}[1]{\left\{{#1}\right\}}
\newcommand{\train}{\mathrm{train}}
\newcommand{\valid}{\mathrm{valid}}
\newcommand{\dd}{\mathrm{d}}

\newcommand{\Dtr}{\ensuremath{D^{\rm TR}}}
\newcommand{\Dva}{\ensuremath{D^{\rm V}}}
\newcommand{\Ntr}{\ensuremath{N^{\rm TR}}}
\newcommand{\Nva}{\ensuremath{N^{\rm V}}}

\newcommand{\thetamax}[1]{\ensuremath{\hat{\theta}^{\max}_{#1}}}
\newcommand{\LCV}[1]{\ensuremath{L_{CV}}(#1)}
\newcommand{\LCVk}[1]{\ensuremath{L^{(k)}_{CV}}(#1)}
\newcommand{\LPPC}[1]{\ensuremath{L_{PPC}}(#1)}
\newcommand{\LPPCk}[1]{\ensuremath{L^{(k)}_{PPC}}(#1)}

\newcommand{\Tkplus}{\ensuremath{T_k^{(+)}}}
\newcommand{\Tkminus}{\ensuremath{T_k^{(-)}}}

\newcommand{\eqn}[1]{eq.~(\ref{eq:#1})}
%\newcommand{\eqns}[1,2]{eqs.~(\ref{eq:#1}) and (\ref{eq:#2})}
\newcommand{\fig}[1]{Fig.~\ref{fig:#1}}
\newcommand{\paper}{document}



\title{A Note on Model Testing}
\author{Jacob T. VanderPlas and David W. Hogg}

\begin{document}
\maketitle

\section{Introduction}

This is a quick note about the relationship between several approaches to
model selection based on data.  We will explore cross-validation (CV),
\highlight{predictive probability criterion (PPC)}, and the Bayes integral
(i.e.~Bayesian evidence).  Our assumptions are as follows:
\begin{enumerate}
  \item Maximum Likelihood is easy to compute
  \item We can compute any posterior through sampling
  \item Though we can sample the prior, we cannot sample it finely enough
    to predictably ``hit'' the full-data likelihood.
\end{enumerate}
These assumptions mean that cross-validation and other maximum likelihood-based
approaches are generally computationally feasible, while approaches based
on Bayesianism are generally much harder.

We will define our notation and outline these three approaches below, and
then show under which assumptions the three are related.

\subsection{Notation}
\begin{itemize}
  \item $I$ is the prior information on $M$.
  \item $M$ is the model which has a vector of model parameters $\theta_M$.
    When we compare multiple models, these will be denoted $M_1$, $M_2$, ...
    with parameters $\theta_{M1}$, $\theta_{M2}$, ...
  \item $D$ is the observed data.  We will divide this data into multiple
    training samples $T_k$ and validation samples $S_k$, which satisfy
    $T_k \cup S_k = D$ and $T_k \cap S_k = \{\}$.
\end{itemize}


\subsection{Cross-validation (CV)}
Cross validation is a well-known technique for model comparison.  Using
maximum likelihood and the above notation, the training model is the
maximum likelihood estimate (MLE) for the training data,
\begin{equation}
  \thetamax{M,k} \equiv \arg\max_\theta \prod_{n \in T_k} P(d_n|\theta, M, I)
\end{equation}
and the cross-validation score is
\begin{equation}
  \label{eq:LCVk}
  \LCVk{M} \equiv P(S_k | \thetamax{M,k}, M, I).
\end{equation}
In the case of multiple ($K$-fold) cross-validation, the total
cross-validation score is
\begin{equation}
  \label{eq:LCV}
  \LCV{M} \equiv \prod_{k=1}^K \LCVk{M}.
\end{equation}
The model with the highest cross-validation score is said to be the
best model.

\subsection{\highlight{Predictive Probability Criterion (PPC)}}
The PPC is an attempt to cache the CV procedure in more formal, Bayesian
terms.  Each cross-validation score $L^{(k)}_{CV}(M)$ is really trying to
get at the quantity
\begin{equation}
  \label{eq:LPPCk}
  \LPPCk{M} \equiv P(S_k|T_k, M, I).
\end{equation}
This can be thought of as a Bayesian approach to single cross-validation.
One might be tempted, in analogy with CV, to define
\begin{equation}
  \label{eq:LPPC}
  \LPPC{M} \equiv \prod_{k=1}^K \LPPCk{M}
\end{equation}
as a global measure of the likelihood using $K$-fold PPC:
unfortunately, this is not well-motivated within Bayesianism
as the individual probabilities in the product are not independent.

\subsection{Posterior Model Probability}
In a Bayesian setting, model selection is accomplished through computing
the posterior model probability
\begin{eqnarray}
  P(M|D,I) &\propto& P(D|M,I)P(M|I)\\
           &\propto& \int \dd\theta_M P(D|\theta_M,M,I)P(\theta_M|M,I) P(M|I)
\end{eqnarray}
This is the fully Bayesian approach, and requires computing a potentially
high-dimensional integral over the parameter space, as well as specifying
priors on the model $M$ and model parameters $\theta_M$.

The choice between two models is accomplished through the use of the odds
ratio,
\begin{eqnarray}
  O_{21} &\equiv& \frac{P(M_2|D,I)}{P(M_1|D,I)}\\
        &=& \frac{P(D|M_2,I)}{P(D|M_1,I)}\ \ \frac{P(M_2|I)}{P(M_1|I)}.
\end{eqnarray}
Often the final term is assumed to be unity, which is a statement
that the models are equally likely {\it a priori}.  In this case, the
odds ratio is given simply by the ratio of the global data likelihood.

As one final note, we can rewrite the Bayes integral $P(D|M,I)$ in a way
that will become useful later.  We will order the data with index $n$ and
define subsets $D_{n} = \{d_1, d_2 ... d_{n-1}, d_n\}$, such that $D_0 = \{\}$
and $D_N = D$.  With this notation we can compactly write the Bayes integral
in terms of a chain of probabilities:
\begin{equation}
  \label{eq:bayes_chain}
  P(D|M,I) = \prod_{n=1}^N P(d_n|D_{n-1},M,I)
\end{equation}

\section{Relationship Between the Approaches}
We have briefly outlined three approaches to model selection based on model
selection.  They range from CV, which is a computationally straightforward
and statistically well-founded procedure, through the fully Bayesian approach
which is in some senses the most rigorous, but has philisophical as well
as computational difficulties.  Nevertheless, the three approaches are similar
in that they assume the data has {\it predictive power}, and use subsets of
the data to evaluate the prediction given other subsets.

In this section we will derive the explicit relationship between the three
approaches, and outline the assumptions under which the three give the same
results.

\subsection{CV and PPC}
The single-fold CV \& PPC scores given by %\eqns{LCVk, LPPCk}
\eqn{LCVk} and \eqn{LPPCk}
have a similar form, and can be related as follows:
\begin{eqnarray}
  \LPPCk{M} &=& P(S_k|T_k,M,I)\\
            &=& \frac{P(S_k,T_k|M,I)}{P(T_k|M,I)}\\
            &=& \frac{\int \dd\theta_M P(S_k,T_k,\theta_M|M,I)}
                     {\int \dd\theta_M P(T_k,\theta_M|M,I)}\\
            &=& \frac{\int \dd\theta_M P(S_k|T_k,\theta_M,M,I)
                                       P(\theta_M|T_k,M,I)P(T_k|M,I)}
                     {\int \dd\theta_M P(\theta_M|T_k,M,I) P(T_k|M,I)}\\
            &=& \frac{\int \dd\theta_M P(S_k|\theta_M,M,I)P(\theta_M|T_k,M,I)}
                     {\int \dd\theta_M P(\theta_M|T_k,M,I)}
\end{eqnarray}
where the final line follows from the assumption that $S_k$ and $T_k$ are
independent given $\theta_M$.  If we make the assumption that
\begin{equation}
  P(\theta_M|T_k,M,I) \approx \delta(\theta_M - \thetamax{M,k})
\end{equation}
then the integrals collapse and we are left with
\begin{equation}
  \LPPCk{M} \approx \LCVk{M}
\end{equation}
This delta function approximation is well-motivated only under two conditions:
\begin{enumerate}
  \item The training sample $T_k$ and validation sample $S_k$ are
    statistically similar, and
  \item The model is much more tightly constrained by $T_k$ than by $S_k$.
\end{enumerate}
If these two assumptions are met, then the single-fold CV score \LCVk{M} is
approximately equivalent to the PPC score \LPPCk{M}.

This now gives us motivation for the definition of \LPPC{M} in \eqn{LPPC}.
If the above assumptions are met for all $k$, then we have
\begin{equation}
  \LPPC{M} \approx \LCVk{M},
\end{equation}
even thought the definition of \LPPC{M} is not well-motivated within the
Bayesian framework.

\subsection{PPC and the Bayes Integral}
Here we will show that the $K$-fold PPC expression \LPPC{M} (\eqn{LPPC}),
though it is a strange expression in a Bayesian context, approximates the
Bayes integral $P(D|M,I)$ when certain assumptions are met.
To accomplish this we will define two more sets
\begin{eqnarray}
  \Tkplus &\equiv& \{d_n\}_{n=1}^{k-1}\\
  \Tkminus &\equiv& \{d_n\}_{n=k+1}^N.
\end{eqnarray}
Notice that with this definition, $\Tkminus = D_{k-1}$ as defined above.
We'll assume a leave-one-out cross-validation scheme, where $S_k =\{d_k\}$
and $T_k = \Tkminus \cup \Tkplus$, for $1 \le k \le K$.  With this we
have
\begin{eqnarray}
  \LCV{M} \approx \LPPC{M} &=& \prod_{k=1}^K P(d_k|\Tkminus,\Tkplus,M,I)\\
         &=& \prod_{k-1}^K \frac{P(\Tkplus|d_k,\Tkminus,M,I) P(d_k|\Tkminus,M,I)}
                               {P(\Tkplus|\Tkminus,M,I)}\\
         &=& P(D|M,I) \prod_{k=1}^K \frac{P(\Tkplus|d_k,\Tkminus,M,I)}
                                        {P(\Tkplus|\Tkminus,M,I)}
\end{eqnarray}
where the last line follows from \eqn{bayes_chain}.
If the quotient on the right hand side is aproximately unity, then we will
have
\begin{equation}
  \LCV{M} \approx \LPPC{M} \approx P(D|M,I).
\end{equation}
This quotient being unity is an interesting approximation: what it says is that
given a validation set \Tkminus{}, the addition of a new observation $d_k$
contributes very little to the predictive power of the model.  This is likely
a valid approximation for $k \gg 1$ where the model is already well constrained
by the $k-1$ previous data, but for $k \sim 1$ the approximation is less
likely to be correct.  It will break down
{\it unless the prior is very informative compared to a single data point}.
For such an informative prior, the addition of a single data point will
have a negligible effect on the prediction, and allow the contribution to the
quotient to be close to 1.

\subsection{Summary}
We have shown the assumptions under which three common model selection criteria
are approximately equivalent.  The 1-fold cross-validation score \LCVk{M}
is approximately equivalent to the predictive probability criterion
\LPPCk{M} if
\begin{enumerate}
  \item The training sample $T_k$ and validation sample $S_k$ are
    statistically similar, and
  \item The model is much more tightly constrained by $T_k$ than by $S_k$.
\end{enumerate}
In the common case that $S_k$ contains a single point, these criteria are
met if the point comprising $S_k$ is not an outlier.

The $K$-fold predictive probability criterion \LPPC{M} is equivalent to
the Bayes integral $P(D|M,I)$ if
\begin{enumerate}
  \item[3] The constraint from a single point is inconsequential compared
    to the constraint from the model prior.
\end{enumerate}
If these three conditions are met, then $K$-fold leave-one-out cross-validation
will give approximately the same results as the Bayesian odds ratio.


\end{document}
